\documentclass[11pt,a4paper,titlepage,final,table]{article}

% Meta-Commands
% !TEX root = "./main.tex"
% Meta-Daten, die mehrfach verwendet werden können
% Angaben Autor*in
\newcommand{\AuthorFirstName}{Vanessa}
\newcommand{\AuthorLastName}{S.}
\newcommand{\AuthorStudentNumber}{1234567}
\newcommand{\AuthorStreet}{Musterstraße 1}
\newcommand{\AuthorZipCode}{12345}
\newcommand{\AuthorLocality}{Musterstadt}
\newcommand{\AuthorEmail}{max.mustermann@smail.inf.h-brs.de}

\newcommand{\SecondAuthorFirstName}{Sven}
\newcommand{\SecondAuthorLastName}{L.}

\newcommand{\ThirdAuthorFirstName}{Hoan}
\newcommand{\ThirdAuthorLastName}{V.}

% Kombinationen
\newcommand{\Author}{\AuthorFirstName\ \AuthorLastName}
\newcommand{\SecondAuthor}{\SecondAuthorFirstName\ \SecondAuthorLastName}
\newcommand{\ThirdAuthor}{\ThirdAuthorFirstName\ \ThirdAuthorLastName}
\newcommand{\AuthorAddress}{\AuthorStreet \newline \AuthorZipCode\ \AuthorLocality}
% Für externe Arbeiten
\newcommand{\ExternalCompany}{Musterfirma}
\newcommand{\Location}{Musterstadt}
% Angaben zum Studium
\newcommand{\CourseOfStudies}{Visual Computing and Games Technology (MSc)}
\newcommand{\CourseOfStudiesDegree}{Bachelor / Master}
% Angaben zum Dokument
\newcommand{\DocumentType}{Whitepaper}
\newcommand{\DocumentTitle}{Tell me mAI story}
\newcommand{\DocumentSubject}{\DocumentType im \CourseOfStudiesDegree\-Studiengang \CourseOfStudies}
\newcommand{\Keywords}{}
\newcommand{\FirstSupervisor}{Prof. Dr. Wolfgang Heiden}
\newcommand{\SecondSupervisor}{Dr. Mirko Seithe}
% Entweder festes Datum oder \today
\newcommand{\DateOfSubmission}{\today}
% Eigentlich obsolet, aber vielleicht kann man das ja auch für andere Fachbereiche benutzen.
\newcommand{\Department}{Computer Science}
\newcommand{\DepartmentGER}{Informatik}
  
% STY-Datei mit Package- und Command-Definitionen   		
\usepackage{./hbrs-inf}


\begin{document}
\maketitle
\clearpage
\pagenumbering{roman}
\setcounter{page}{2}
\pagestyle{plain}
\parskip 0.5em

% Inhaltsverzeichnis
%\setcounter{tocdepth}{2}
%\tableofcontents
%\clearpage

% Abbildungsverzeichnis
%\listoffigures
%\clearpage

% Tabellenverzeichnis
%\listoftables
%\clearpage


%\clearpage
\pagenumbering{arabic}

\lstset{
	basicstyle=\ttfamily\small,
	keywordstyle=\bfseries,
	language=[Sharp]C
}

% Insert sections here

% TODO Does the proposal need an Abstract or a Management Summary?

\section{Introduction}
In this chapter, goals of the research project are described and contributors are introduced.
\subsection{Goals}

\subsection{Contributors}

\begin{itemize}
    \item team members with their qualifications
\end{itemize}


\section{Related Work}
In the first sub section of this chapter, an overview of the literature in our field and context is provided.
In the second sub section of this chapter, our previous studies are described.
\subsection{Literature}

\subsection{Previous Studies}

\begin{itemize}
    \item our own related work 
    \item Exploration of voice interfaces in lecture Advanced UI
\end{itemize}


\section{Planned Technology and Artistic Concept}
In this chapter, our concepts are described along with potential research questions.
Moreover, technical risks are considered, whereby we also state our solutions to mitigate the risks.
\subsection{System Design}


\begin{itemize}
    \item how is the system organized
    \item diagram(s) with description
\end{itemize}


\subsection{Technical Development}

\begin{itemize}
    \item what do we want to develop                                   
    \item regarding 1st party software 
    \item some indication regarding non-functional requirements, e.g. code should run with and without GPU
    \item aspects of software craftsmanship or clean code or code quality
\end{itemize}

\subsubsection{Fine-tuning of language models for story generation}

Regarding the task of generating a story, we want to look into methods of fine-tuning pre-trained language models.
The idea is to leverage the full knowledge of an existing, freely available, pre-trained language model, which acquired its knowledge during its elaborate training process, and to tailor it towards a particular task, to obtain more apt outputs.

In our case, we want a fine-tuned language model to generate better stories for our target audience, the children.
A qualification or quantification of what better means might be part of the research, but in the most simple and straight forward approach, we generate multiple stories and judge as humans as to how good and suitable these stories are, thereby comparing them with previously generated stories.

Fine-tuning language models can be especially valuable when an input dataset is limited, because training from scratch with a limited dataset usually yields a model that has not seen enough, that does not generalize well and is quickly over-fitted on the limited dataset, that does not achieve the task and instead yields rather random outputs.

Two key aspects in fine-tuning are going to be the choice of the pre-trained model and the given dataset.

\subsection{Technical Risks}


\begin{itemize}
    \item name and describe each considered technical risk 
    \item also for each technical risk show how we want to mitigate it
\end{itemize}

\subsubsection{Runtime efficiency and size of language models}

In fine-tuning or in inference, a language model may turn out to be inefficient and slow.
This could be a problem in particular in inference, since then a user would have to wait longer until a story is generated.

Or, a language model may simply be too large to fit into the computer memory, which would prevent it from being used at all on the particular computer.

To mitigate these two technical risks, we intend to look into small language models (SLMs) and their applicability to our goal of generating stories suitable for children.
In particular also, we intend to look into optimized language models that consist of and calculate with integer numbers instead of floating point numbers.
This is a technological trend that allows models of a particular size in bytes to consist of more or wider layers, thus have more weights, more parameters, allow for lager context sizes.
In other words, by changing the internal numerical data type, researchers have been able to create larger models that still are of feasible sizes in bytes and thus suitable for local execution on consumer hardware.

A different mitigation to this technical risk, which is however more difficult and a less readily accessible, could be the use of streaming, which is a technological trend where a model is at no point in time fully loaded into the computer memory and is instead streamed from a local disk or from some other kind of storage.
A possible pattern could be to stream such a model, during a forward-pass or during back-propagation, in batches of layers, so that at each point in time only a subset, in other words a batch, of layers of the model would need to be present in computer memory, thus overcoming the memory bottleneck.

\subsubsection{Dataset for fine-tuning language models}

To obtain or to curate a dataset suitable for fine-tuning a pre-trained language model towards generating better stories for children may turn out to be a hard task.



\subsection{Potential Research Questions}

\begin{itemize}
    \item formulation of research questions 
    \item coined regarding technology
    \item coined regarding use cases in a scientific sense
    \item targeting the gaining of scientific insights
\end{itemize}


\subsection{Development Goals}


\section{Materials and Methods}
In this chapter, required hardware and applied methods are discussed.
\subsection{Materials}

\subsection{Methods}

\begin{itemize}
    \item methods
    \item e.g. questionnaire
    \item e.g. metrics
    \item e.g. certain benchmarks
\end{itemize}



\section{Project Planning}
In this chapter, our initial project plan is provided, which applies at the beginning of the project and which is binding.
\subsection{Work Packages}


\begin{itemize}
    \item our description of our work packages
    \item is of the latest update including the work packages regarding project management
\end{itemize}


\subsection{Milestones}

\subsection{Responsibilities}

\begin{itemize}
    \item roles with a description of the associated responsibilities
    \item define which team members assume which roles
\end{itemize}


\newpage

%Literaturverzeichnis
\begin{thebibliography}{2}
\bibitem{1}
Polsley et al. https://link.springer.com/article/10.1007/s40593-021-00279-7, 2021
\bibitem{2}
Beltzung et al https://www.ncbi.nlm.nih.gov/pmc/articles/PMC9945213/, 2023
\bibitem{3}
Google Creative Labs. https://experiments.withgoogle.com/quick-draw, 2017
\bibitem{4}
Meta. https://about.fb.com/news/2021/12/using-ai-to-animate-childrens-drawings/, 2021
\bibitem{5}
Shakeri. https://dl.acm.org/doi/abs/10.1145/3462204.3481771 , 2021
\bibitem{6}
C. Zhang et al. https://dl.acm.org/doi/abs/10.1145/3491102.3501914, 2022
\bibitem{7}
Z. Zhang et al. https://dl.acm.org/doi/abs/10.1145/3491102.3517479, 2022
\bibitem{8}R. Brunelli. Template Matching Techniques in Computer Vision: Theory and Practice, Wiley, ISBN 978-0-470-51706-2, 2009
\bibitem{9}J. Canny. A Computational Approach To Edge Detection, IEEE Transactions on Pattern Analysis and Machine Intelligence, 1986
\bibitem{10}
Girshick, R., Donahue, J., Darrell, T., and Malik, J. Rich feature hierarchies
for accurate object detection and semantic segmentation. CVPR, 2014
\end{thebibliography}

\end{document}
