%!TEX root="../main.tex"
\section{Related Works}

%\subsection{Inspriation}

\subsection{Scientific Research}
Children's drawings have been a research topic for many years with the intention to analyze these drawings. More and more these drawings are analyzed with the help of machine learning or deep learning algorithms [\cite{1}, \cite{2}]. Separately, the detection of drawings are used in a more gamified way as well. In 2017 Google introduced its machine learning based game “Quick, Draw!”, where you draw a sketch and the neural network tries to guess what it is. At the same time the game uses the sketches to learn and identify more sketches [\cite{3}]. Meta went even further and uses an AI to animate children’s drawings when they have arms and legs. These characters can be animated to do different poses like dancing or jumping as well as being set in different backgrounds [\cite{4}]. 
Using artificial intelligence for creative storytelling in general [\cite{5}] and children's stories specifically is not a new approach, but has gained a lot of attention in the last few years. Zhang et al. [\cite{6}] developed a child - AI collaborative drawing system for story telling called “StoryDrawer” . Another example is the human - AI collaborative chatbot “StoryBuddy” that enables children and parents to use an AI for storytelling [\cite{7}]. Our research project aims to bring a new idea to this research area by using a mixed media approach with paper drawings influenced with tangibles that create an enjoyable children's story. \par

AI techniques have been increasingly expanding and bridging its application to diverse field such as Computer Vision, which unlock the understanding of visual artistic expression, e.g., in children’s drawing. The domain of object detection has advanced from early algorithms like edge-based methods (Canny edge detection [\cite{9}]) or template matching [\cite{8}], to modern deep learning methods like CNN (Convolutional Neural Networks [\cite{10}]). Additionally, the convergence of generative AI model and object detection has enabled the potential of collectively textual and visual information processing, to provide robust solutions for visual scene understanding.
