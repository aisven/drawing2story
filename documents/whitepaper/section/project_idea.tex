%!TEX root="../main.tex"
\section{Project Idea}
\subsection{Vision}
Imagine children could present a drawing of animals to our AI to make it tell a children’s story involving the animals as characters. At the heart of the interaction are the children’s drawings, digitized as images. Easy to grasp features allow children to influence the story. They can provide clues to the AI regarding character importance, relations, activities. Thus, children interact with and learn about AI in simple creative ways while having fun.

\subsection{Core Features}
Interactive storytelling tool for elementary school children aged 6 to 8 as well as young secondary school children aged 8-12. Its unique capability lies in its ability to understand hand-drawn animal illustrations, which should be selected from a pre-defined animal catalog, and in response, generate captivating stories involving the recognized characters. \par
With the help of numbers, markers, and activity cards, the children can influence the story that will be created.
\begin{itemize}
    \item Using numbers, the importance of the characters can be denoted, for example a 9 would be a main character.
    \item Markers are used to indicate which characters should meet with and talk to which other characters during the story.
    \item Pre-defined activity cards can be placed onto the picture to indicate which characters should take part in which activities (e.g. eating, drinking, jumping, going for a walk) during the story. (The activity cards are considered a potential feature, still to be prioritized with respect to other features and details.)
  \end{itemize}
Essentially, a processing pipeline transforms an annotated child’s drawing into a generated story, which is then read aloud by the machine or device.
