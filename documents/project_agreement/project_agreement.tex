%laden der Präambel mit Latexbefehlen/-klassen
%\include{expose.preambel}
\documentclass{article}

\usepackage{titling}


\title{Project Agreement - Tell me mAI story}
\author{Vanessa S., Sven L., Hoan V. \\\\ University of Applied Sciences Bonn-Rhein-Sieg \\ Visual Computing and Games Technology (MSc)}

\date{\today}

\begin{document}

\maketitle

This document serves as a compass for the team, outlining the composition, roles, decision-making processes, and regular interactions within our project.

\section{Project title}
\subsection{Working title}
The working title of the project is "Tell me mAI story".
\subsection{Technical project name}
The technical project name is "drawing2story".
It is short and does not contain white space or special characters.
It represents the fact that drawings are converted into stories.
The technical project name is used in file names, artifact names, repository names and source code.

\section{Team composition and role distribution}
\subsection{Team members and roles}
Roles are distributed based on the qualifications of each team member.
The qualifications are listed in the white paper of the project.
\begin{itemize}
    \item Vanessa S. | Project Manager, Designer | project management, cost management, scientific writing, scientific research, interface development, design
    \item Sven L. | AI Engineer, Product Manager | product management, vision keeping, programming, AI research, technical documentation, dataset curation
    \item Hoan V. | AI Engineer, Technical Writer | solution evaluation, programming, AI research, technical documentation, LaTeX infrastructure, dataset curation
\end{itemize}

\subsection{Decisions agreement and project representative}
Decisions are made based on support and rejection feedback obtained for a proposed idea or solution.
For a specific topic, a team member proposes an idea or a solution and gets feedback from the other team members.
If the idea or solution is approved by all team members, it is integrated into the development roadmap.
Otherwise, the idea or solution can be put on hold if at least one team member shows support, or rejected if most team members reject it.
Any idea or solution may be discussed and refined more elaborately by the team members before a decision is made.
Any decision made, in other words, any idea or solution, can be questioned at a later point in time, and, if all team members then agree, it can be reopened for refinement.

Sven L. is the representative of the project.

\subsection{Meeting schedule}
Discord is the main channel of communication.
Additionally, a WhatsApp group has been created and is sometimes used regarding urgent matters or communication of temporary absence.
During the development phase, the team members come together to discuss ideas, solutions, achievements, current events, planned work, in a weekly online meeting.

\section{Handling changes in the team}
\subsection{What happens when someone (temporarily) leaves the team, who takes over the tasks, and what happens to the rights and obligations of the departing member?}
\begin{itemize}
    \item Task Handover Protocol | (temporary) transfer of tasks to other team members.
    \item Rights and Obligations | Each member is entitled to discontinue their involvement in the project without the requirement of specifying a reason.
    \item A leaving member is obligated to announce their departure at least two weeks before the official date. During this time, the departing team member needs to push all development progress and conduct knowledge transfer meetings. All project assets, including files and hardware, are required to be transferred to a designated team member.
\end{itemize}
\subsection{How are new team members welcomed?}
A new member is always welcome, since there are different tasks in scope of the project that require skills and expertise.
If a majority of team members agree, the new member is onboarded.
\subsection{How is accumulated knowledge documented and passed on? How is a handover structured?}
Every member is assigned to work on specific features in each stage of the development.
Any progress made must be pushed to the according working branch on Gitlab along with a comprehensive README.md documentation summarizing the progress, challenges, and a list of remaining tasks.
Best practices in documenting source code and functionality are applied.
Ideally, much of the source code is self-explanatory, while team members are however encouraged to include verbose inline comments to guide other team members and to highlight important aspects right where they are implemented or relevant in the source code.
In terms of knowledge transferring, to either a new onboarded member or from a departing member to the team, a meeting with all involved members is organized.

\section{Intellectual property and copyright}
\subsection{Ownership of the project}
The project is collectively owned by the team members, and all rights and responsibilities are distributed accordingly.
Note that the group of team members is considered a collective, but it does not form an explicit legal entity, i.e. no organization is explicitly legally founded.
However, in the sense of the law, the group of team members may still implicitly, as per default, be regarded as a certain legal entity, whichever the law sees fit.
\subsection{Ownership of created assets, program code}
All rights reserved.
This project is currently without license and all rights are reserved, i.e. the team members retain all rights provided by copyright law unless otherwise noted.
The created assets and program code are the intellectual property of their respective creators and are subject to the specified usage rights within the project.
\subsection{Ownership of exploitation and license rights}
Exploitation and license rights belong to the respective creator of the content and are managed in accordance with the agreed-upon terms of the project.
\subsection{Future work}
Rights to derivative works such as sequels or spinoffs are regulated considering the original copyrights and in consensus with the involved parties
\subsection{Ownership of acquired licenses and assets}
Any external licenses and assets upon team agreement should be acquired using project budget and belong to the project as a whole and should be only used within the project development.
In case any third party commercial license, closed source license, restrictive license, or otherwise invasive license, applies, the team needs to assess possible legal impacts on use, and possible reuse, of project outputs and products.

\section{Financial aspects}
\subsection{Who finances the project?}
No financing is needed.
At some development stage team members probably use computers with powerful GPUs and CPUs as provided to them for free by University of Applied Sciences Bonn-Rhein-Sieg, Computer Science Department.
\subsection{Who receives the income/profits from the project}
Income and profits derived from the project will be equitably distributed among all members involved in the project upon its completion.
\subsection{What happens if investors come on board?}
In the event of investor involvement a new project agreement, in terms of a contract involving team members and investors, has to be written and signed.
Specific terms and impacts on financing and profit sharing will be defined collaboratively with the investors and managed in alignment with the interests of all project stakeholders.
\subsection{Who handles cost management and accounting}
Cost management and accounting will be overseen by the designated project member to ensure transparent and accurate administration of the project's financial aspects.

\section{Conflict resolution}
\subsection{How do we handle issues not explicitly addressed in the contract?}
The process described in section "Decisions agreement and project representative" applies.
In case an issues is not expressly addressed in the contract and not solved via that process, the team will seek mutual clarification to reach an agreement that reflects fairness, collective interests, individual interests.
\subsection{Team Conflict Resolution Strategies}
In the event of conflicts, clear communication and open discussions involving neutral third parties are considered.
\subsection{Procedures in the Case of Disagreements or Disputes}
In the event of disagreements or disputes, four levels of escalation are traversed.
\begin{itemize}
    \item Talk to a person of choice, present results to the whole team, then decide as a team
    \item Meeting with the whole team, then decide as a team
    \item Meeting with the project representative, to resolve the escalated issue
    \item Meeting of the whole team with a responsible Professor, to resolve the escalated issue
\end{itemize}

\vfill
\section*{Bonn, \today}

\begin{minipage}{1\textwidth}
	\centering
	Project Members
	\hrule
\end{minipage}

\end{document}
