%laden der Präambel mit Latexbefehlen/-klassen
%\include{expose.preambel}
\documentclass{article}

\usepackage{titling}


\title{Project Agreement-Tell me mAI story}
\author{Vanessa Skowronsky, Sven Ludwig, Hoan Vu \\\\ University of Applied Sciences Bonn-Rhein-Sieg \\ Visual Computing and Games Technology (MSc)}

\date{\today}

\begin{document}

\maketitle

This document serves as a compass for the team, outlining the composition, roles, decision-making processes, and regular interactions within our project.
\section{Team composition and role distribution}
\subsection{Team members and roles}
Roles are distributed based on the experience level of each member: 
\begin{itemize}
    \item Vanessa Skowronsky: Project management, scientific writing and research, interface development and design
    \item Sven Ludwig: Vision keeping, programming and AI research
    \item Hoan Vu: Programming, general development documenting
\end{itemize}

\subsection{Decisions agreement and project representative}
Decisions are made through support and rejection feedback. For a specific topic, a more experienced person in that field would propose ideas and suggestions, and expect feedback from the other two members. \\

If the proposal gets approved by both members, it will go to the development roadmap. Otherwise, the proposal could be either put on hold (at least one other team member shows support) or rejected (if both other team members reject it).\\

Sven Ludwig is the representative of the project.
\subsection{Meeting schedule}
Mostly through text discussions on the group Discord channel, as everyone has a different working schedule. Further later in development, we will organize a weekly (online or present) meeting.

\section{Handling changes in the team}
\subsection{What happens when someone (temporarily) leaves the team? Who takes over the tasks? What happens to the rights and obligations of the departing member?}
\begin{itemize}
    \item Task Handover Protocol: (temporary) transfer of tasks to other team members. 
    \item Rights and Obligations: Each member is entitled to discontinue their involvement in the project without the requirement of specifying a reason. A leaving member is obligated to announce their departure at least 2 weeks before the official date. During this time, this member needs to push all development progress on the project's Github repositories and conduct (a) knowledge transfer sessions. All project assets, including files and hardware, are required to be transferred to a designated team member.
\end{itemize}

\subsection{How are new team members welcomed?}
New member is always welcomed, there are different tasks in scope of the project that are suitable for everyone's skill and expertise. Final decision should be made through member votes.

\subsection{How is accumulated knowledge documented and passed on? How is a handover structured?}

Every member is assigned to work on specific features in each stage of the development. Any progress made needs to be pushed into the according working branch on Gitlab along with a comprehensive README.md documentation summarizing the progress, challenges, and a list of remaining tasks . \\

In terms of knowledge transferring (to either a new onboarded member or from a departing member to the team), a meeting session with all involved members should be organized.

\section{Intellectual property and copyright}
All rights reserved.
This project is currently without license and all rights are reserved, i.e. the contributors retain all rights provided by copyright law unless otherwise noted.
\subsection{Ownership of the project}
The project is collectively owned by the contributors, and all rights and responsibilities are distributed accordingly.
Note that the group of contributors is considered a collective, but it does not form a explicit legal entity, i.e. no organization is explicitly legally founded.
However, in the sense of the law, the group of contributors may still implicitly, as per default, be regarded as a certain legal entity, whichever the law sees fit.
\subsection{Ownership of created assets, program code}
The created assets and program code are the intellectual property of their respective creators and are subject to the specified usage rights within the project.
\subsection{Ownership of exploitation and license rights}
Exploitation and license rights belong to the respective creator of the content and are managed in accordance with the agreed-upon terms of the project.
\subsection{Future work}
Rights to derivative works such as sequels or spinoffs are regulated considering the original copyrights and in consensus with the involved parties
\subsection{Ownership of acquired licenses and assets}
Any external licenses and assets upon team agreement should be acquired using project budget and belongs to the project as a whole and should be only used within the project development.

\section{Financial aspects}
\subsection{Who finances the project?}
No financing is needed. At some development stage we will be using computers with powerful GPU and CPU which is provided by University of Applied Sciences Bonn-Rhein-Sieg, Computer Science Department
\subsection{Who receives the income/profits from the project}
Income and profits derived from the project will be equitably distributed among all members involved in the project upon its completion (expected in the 3. semester).
\subsection{What happens if investors come on board?}
In the event of investor involvement, specific terms and impacts on financing and profit sharing will be defined collaboratively with the investors and managed in alignment with the interests of all project stakeholders.
\subsection{Who handles cost management and accounting}
Cost management and accounting will be overseen by Vanessa Skowronsky to ensure transparent and accurate administration of the project's financial aspects.

\section{Conflict resolution}
\subsection{How do we handle issues not explicitly addressed in the contract?}
In cases where issues are not expressly addressed in the contract, the team will seek mutual clarification to reach an agreement that reflects fairness and a willingness.

\subsection{Team Conflict Resolution Strategies}
In the event of conflicts, clear communication and open discussions involving neutral third parties are considered.
\subsection{Procedures in the Case of Disagreements or Disputes}
In the event of disagreements or disputes, a structured process will be followed. This includes a factual analysis of the problem, identification of potential solutions. Eventually, neutral third parties are considered

\vfill
\section*{Bonn, \today}

\begin{minipage}{1\textwidth}
	\centering
	Project Members
	\hrule
\end{minipage}

\end{document}

